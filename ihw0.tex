\problemset{Комбинаторика и теория графов}
\problemset{Индивидуальное домашнее задание №0}	% поменяйте номер ИДЗ

\renewcommand*{\proofname}{Решение}

%%%%%%%%%%%%%% ЗАДАНИЕ №1 %%%%%%%%%%%%%%
%% Условие задания №1
\begin{problem}
	Бинарное отношение задано матрицей. С помощью алгоритма Уоршелла найдите его транзитивное замыкание.
	$$ \left( \begin{array}{ccccc}
		1 &1 &0 &0 &1 \\0 &0 &1 &0 &1 \\1 &0 &1 &0 &0 \\0 &0 &0 &1 &1 \\0 &1 &0 &0 &0\end{array} \right) $$
\end{problem}

%% Решение задания №1
\begin{proof}

\end{proof}

%%%%%%%%%%%%%% ЗАДАНИЕ №2 %%%%%%%%%%%%%%
%% Условие задания №2
\begin{problem}
	Найдите а) наименьшее; б) наибольшее возможное количество компонент связности в графе с 20 вершинами и 
	19 рёбрами.
\end{problem}

%% Решение задания №2
\begin{proof}

\end{proof}

%%%%%%%%%%%%%% ЗАДАНИЕ №3 %%%%%%%%%%%%%%
%% Условие задания №3
\begin{problem}
	Условие задачи №3.
\end{problem}

%% Решение задания №3
\begin{proof}
	Решение задачи №3.
\end{proof}